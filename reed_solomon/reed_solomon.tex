\documentclass[a4paper]{article}
%
\usepackage{charter}
\usepackage{parskip}
\usepackage[margin=2cm]{geometry}
%
\title{Reed-Solomon algorithms for QR codes}
\author{Sidney~Cadot}
%
\begin{document}
\maketitle
\section{Introduction}
%
This document provides background information on Reed-Solomon codes.
The aim is to explain the encoding and decoding algorithms used in the information area of QR code symbols.

This document refers extensively to the standard ISO/IEC~18004:2024(en), which is the
most recent edition of the standard that describes QR code symbols.
%
\section{Construction of GF($2^8$)}
The Reed-Solomon codes used in QR codes are \emph{block codes}, where each code word is a
finite-length vector of code symbols. The code symbols are elements of the Galois field of
order $2^8=256$.

The precise definition of a Galois field can be found in any introductory textbook
on algebra. For the purpose of this document, it suffices to think of it as a set of
256 elements with well-defined operation for addition and multiplication that behave
very similar to the operations defined on the numbers we normally use.

For example, there is an element zero (which is the additive identity) and an element
one (which is the multiplicative identity); subtraction is well defined although a bit
strange ($x-y$ is equal to $x+y$), and division by any non-zero element is well-defined,
yielding one of the 256 elements. Division by the element 0 is undefined.
Also, the normal distributive, commutative, and associative laws hold.

Apart from the elements $0$ and $1$, there is also an element called the primitive root,
which is denoted as $\alpha$. The powers $a^k$ for $k$ from $0$ to $254$ generate the
non-zero elements of GF($2^8$). In particular, $\alpha^0=1$.

The 256 elements of $GF^8$ can thus be denoted as $\{ 0, \alpha^0, \alpha^1, \ldots, \alpha^{254} \}$.
Expanding the sequence of powers of $\alpha$ further repeats the non-zero values, i.e., $\alpha^{255}=\alpha^0=1$, $\alpha^{256}=\alpha^2$, and so on.

There is essentially (`up to isomorphism') only one field with precisely 256 elements, which is GF($2^8$).

%It can be constructed (meaning, by assuming a property
%of $\alpha$, specifically, that it is a root of a specific polynomial over the to-be-constructed 
%The primitive root $\alpha$

\section{Generator polynomials used in QR codes}

\begin{displaymath}
G_L(x) = \prod_{i=0}^{L-1}(x-\alpha^i)
\end{displaymath}

\section{Encoding}

Property we're looking for:

\begin{displaymath}
D(x) \cdot x^L + C(x) \equiv 0 \pmod{G_L(x)}
\end{displaymath}

Calculating $C(x)$:

\begin{displaymath}
C(x) \equiv -(D(x) \cdot x^L) \equiv D(x) \cdot x^L \pmod{G_L(x)}
\end{displaymath}

\section{Decoding and error correction}
\end{document}
